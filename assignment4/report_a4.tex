\documentclass{article}\usepackage[]{graphicx}\usepackage[]{color}
%% maxwidth is the original width if it is less than linewidth
%% otherwise use linewidth (to make sure the graphics do not exceed the margin)
\makeatletter
\def\maxwidth{ %
  \ifdim\Gin@nat@width>\linewidth
    \linewidth
  \else
    \Gin@nat@width
  \fi
}
\makeatother

\definecolor{fgcolor}{rgb}{0.345, 0.345, 0.345}
\newcommand{\hlnum}[1]{\textcolor[rgb]{0.686,0.059,0.569}{#1}}%
\newcommand{\hlstr}[1]{\textcolor[rgb]{0.192,0.494,0.8}{#1}}%
\newcommand{\hlcom}[1]{\textcolor[rgb]{0.678,0.584,0.686}{\textit{#1}}}%
\newcommand{\hlopt}[1]{\textcolor[rgb]{0,0,0}{#1}}%
\newcommand{\hlstd}[1]{\textcolor[rgb]{0.345,0.345,0.345}{#1}}%
\newcommand{\hlkwa}[1]{\textcolor[rgb]{0.161,0.373,0.58}{\textbf{#1}}}%
\newcommand{\hlkwb}[1]{\textcolor[rgb]{0.69,0.353,0.396}{#1}}%
\newcommand{\hlkwc}[1]{\textcolor[rgb]{0.333,0.667,0.333}{#1}}%
\newcommand{\hlkwd}[1]{\textcolor[rgb]{0.737,0.353,0.396}{\textbf{#1}}}%

\usepackage{framed}
\makeatletter
\newenvironment{kframe}{%
 \def\at@end@of@kframe{}%
 \ifinner\ifhmode%
  \def\at@end@of@kframe{\end{minipage}}%
  \begin{minipage}{\columnwidth}%
 \fi\fi%
 \def\FrameCommand##1{\hskip\@totalleftmargin \hskip-\fboxsep
 \colorbox{shadecolor}{##1}\hskip-\fboxsep
     % There is no \\@totalrightmargin, so:
     \hskip-\linewidth \hskip-\@totalleftmargin \hskip\columnwidth}%
 \MakeFramed {\advance\hsize-\width
   \@totalleftmargin\z@ \linewidth\hsize
   \@setminipage}}%
 {\par\unskip\endMakeFramed%
 \at@end@of@kframe}
\makeatother

\definecolor{shadecolor}{rgb}{.97, .97, .97}
\definecolor{messagecolor}{rgb}{0, 0, 0}
\definecolor{warningcolor}{rgb}{1, 0, 1}
\definecolor{errorcolor}{rgb}{1, 0, 0}
\newenvironment{knitrout}{}{} % an empty environment to be redefined in TeX

\usepackage{alltt}
\usepackage[letterpaper, portrait, margin=1in]{geometry}
\usepackage{amsmath}
\usepackage{hyperref}

\title{MATH 680 - Assignment \#4}
\author{Kevin McGregor}
\date{November 30th, 2015}

\newcommand{\bb}{\bar{\beta}}
\IfFileExists{upquote.sty}{\usepackage{upquote}}{}
\begin{document}
\maketitle

\section*{Question 1}
\subsection*{(a)}
In this problem we have $f(\beta)=\|y-X\beta\|^2$, $g(\beta)=R^\top\beta$, $R=(0,0,\dots,1)^\top$.  As seen in earlier written assignments, $\nabla f(\beta)=-2X^\top y + 2X^\top X\beta$.  Similarly, $\nabla g(\beta)=R$.  So we wish to find $(\bar{\beta},u)$ such that:
\begin{eqnarray*}
  -X^\top y + X^\top X\bb + \frac{u}{2}R &=& 0 \\
  R^\top \bb &\leq & 0 \\
  uR^\top \bb &=& 0 \\
  u &\geq & 0.
\end{eqnarray*}
We then have that $\bb=(X^\top X)^{-1}(X^\top y -\frac{u}{2}R)$. Plugging this into the third KKT condition gives,
\begin{eqnarray*}
  uR^\top (X^\top X)^{-1}(X^\top y -\frac{u}{2}R) &=& 0 \\
  u &=& \frac{2R^\top (X^\top X)^{-1}(X^\top y)}{R^\top(X^\top X)^{-1}R},
\end{eqnarray*}
provided $2R^\top (X^\top X)^{-1}(X^\top y)>0$.  Otherwise, we have $u=0$.  So, if OLS finds a negative value of $\beta_p$, then nothing more needs to be done.  If OLS finds a positive value for $\beta_p$, then choosing $u$ as stated above will ensure the second KKT condition.

\subsection*{(b)}
\emph{The R function for this is called fitBetaNeg, and is found in the file fitBetaNeg.R.  Testing for parts (b) and (c) is found in the file a4\_q1.R}.  The KKT point is $$(\bb,u)=\left(\begin{bmatrix}4.9 & 4.1 & 2.5 & 2.4 & 0\end{bmatrix}^\top,122.6177\right)$$.  The KKT conditions are met at this point.

\subsection*{(c)}
The KKT point is $$(\bb,u)=\left(\begin{bmatrix}3.6 & 3.3 & 2.1 & 1.6 & -1.9\end{bmatrix}^\top,0\right)$$.  The KKT conditions are met at this point.

\section*{Question 2}
\subsection*{(a)}
We'll take our proposal distribution to be:
\begin{eqnarray*}
  g(x) &=& \frac{\beta^a x^{a-1}}{\Gamma(a)}e^{-x},
\end{eqnarray*}
where $a=\lfloor\alpha\rfloor$.  Then we have that,

\begin{eqnarray*}
  \frac{f(x)}{g(x)} &=& \frac{\beta^{\alpha-a} x^{\alpha-a}}{\Gamma(\alpha)}\Gamma(a) e^{-(\beta-1)x} \\
  &\propto&  x^{\alpha-a}e^{-(\beta-1)x},
\end{eqnarray*}
and so,
\begin{eqnarray*}
  \nabla \frac{f(x)}{g(x)} &\propto& (\alpha-a)x^{\alpha-a-1}e^{-(\beta-1)x}-x^{\alpha-a}(\beta-1)e^{-(\beta-1)x}.  
\end{eqnarray*}
Setting the derivative to zero results in $x=\frac{\alpha-a}{\beta-1}$.  We can then conclude that:
\begin{eqnarray*}
  \frac{f(x)}{g(x)} &\leq& \frac{\beta^{\alpha-a}}{\Gamma(\alpha)}\left(\frac{\alpha-a}{\beta-1}\right)^{\alpha-a} \Gamma(a) e^{-(\alpha-a)} \\
  &=& c.
\end{eqnarray*}

So, we just draw a sample of size $\lfloor\alpha\rfloor$ from $Exp(1)$, using the inverse distribution function $-\log(1-x)$.  The sum of these random variables, $Z$, will follow $Gamma(\lfloor\alpha\rfloor,1)$.  Then we draw $U \sim Unif(0,1)$ and accept if $U<\frac{f(Z)}{cg(Z)}$.  This will generate a sample from $Gamma(\alpha,1)$.  Then, multiply the observation by $1/\beta$ to get $Gamma(\alpha,\beta)$.

\subsection*{(b)}
\emph{See R function gammaSamp.R}.

\subsection*{(c)}
\emph{The code to set up the simulation is found in the R file a4\_q2c.R}.  I ran the algorithm with $\alpha \in \{ 1.5, 5.3, 10.9, 50.2, 75.2 \}$, and $\beta=2$.  The estimated acceptance probabilites are found in Table~\ref{tab:probs}.

\begin{table}[ht]
\centering
\begin{tabular}{ll}
  \hline
 $\alpha$ & Prob \\ 
  \hline
1.5 & 0.90 \\ 
5.3 & 0.65 \\ 
10.9 & 0.66 \\ 
50.2 & 0.03 \\ 
75.2 & 0.01 \\ 
   \hline
\end{tabular}
\caption{Estimated acceptance probabilities for various values of $\alpha$.}
\label{tab:probs}
\end{table}

\section*{Question 3}
\subsection*{(a)}
\begin{eqnarray*}
  f(\theta | x) &=& \frac{f(x|\theta)f(\theta)}{f(x)} \\
  &=& \frac{\frac{1}{\sqrt{2\pi}} e^{-\frac{(x-\theta)^2}{2}} \frac{1}{\pi(1+\theta^2)}}{\int\frac{1}{\sqrt{2\pi}} e^{-\frac{(x-\theta)^2}{2}} \frac{1}{\pi(1+\theta^2)} d\theta } \\
  &\propto& e^{-\frac{(x-\theta)^2}{2}} \frac{1}{\pi(1+\theta^2)}.
\end{eqnarray*}

\subsection*{(b)}
Let $$g(\theta)=\frac{1}{\pi s\left(1+\left(\frac{\theta-m}{s}\right)\right)}$$. 
We have that, 
\begin{eqnarray*}
\frac{\tilde{f}(\theta)}{g(\theta)} &=& e^{-\frac{(x-\theta)^2}{2}} \frac{s\left(1+\left(\frac{\theta-m}{s}\right)\right)}{1+\theta^2}
\end{eqnarray*}
If we choose $m=0$ and $s=1$, then we have that:
\begin{eqnarray*}
\frac{\tilde{f}(\theta)}{g(\theta)} &\leq& 1
\end{eqnarray*}

So the algorithm proceeds as follows:
\begin{enumerate}
  \item Draw a sample, Z, from $Cauchy(0,1)$
  \item If $U<\frac{\tilde{f}(\theta)}{g(\theta)}$, where $U\sim Unif(0,1)$, then accept the sample.  Reject otherwise.
\end{enumerate}

\subsection*{(c)}
\emph{The function is called normCauchyPrior, and it is found in the file normCauchyPrior.R.  I also ran a simulation to test the function, found in a4\_q3.R}.  I take a sample of size 100 from this posterior distribution and calculate the expected value.  I replicated this 1000 times, and the mean of the expected values was $0.5570$, and the variance, $0.0054$.

\subsection*{(d)}
Since we do not know the normalizing constant, we need to use weighted importance sampling. So, if we take a random sample $Z_i$, $i=1,\dots,n$ from $g$, this estimator takes the form:
\begin{eqnarray*}
  \hat{E}(\theta | X=x) &=& \frac{\sum_{i=1}^n z_i e^{-\frac{(z_i-x)^2}{2}}}{\sum_{i=1}^n e^{-\frac{(z_i-x)^2}{2}}}
\end{eqnarray*}

\subsection*{(e)}
\emph{The function is called importNormCauchy, and is found in the file normCauchyPrior.R.  Testing for this function is found in a4\_q3.R}. Using the estimator from part (d), and running 1000 replications, we get the mean of the estimator to be $0.5013$ and a smaller variance of $0.0042$.

\section*{Question 4}
\subsection*{(a)}
\begin{eqnarray*}
  E(\hat{E}_1) &=& E\left[ \frac{1}{T}\sum_{i=1}^T h(X_i) \right] \\
  &=& \frac{1}{T}\sum_{i=1}^T E\left[h(X)\right] \\
  &=& E\left[h(X)\right]
\end{eqnarray*}

\begin{eqnarray*}
  E(\hat{E}_2) &=& E\left[ \frac{1}{n-T}\sum_{i=1}^{n-T} \frac{(c-1)h(R_i)f(R_i)}{cg(R_i)-f(R-i)} \right] \\
  &=& \frac{1}{n-T}\sum_{i=1}^{n-T} \left[ (c-1)E\left[\frac{h(R_i)f(R_i)}{cg(R_i)-f(R-i)}\right] \right]
\end{eqnarray*}
Now, the density of the rejected samples is $K(cg(R)-f(R))$, and so, $$ K=\frac{1}{\int cg(r)-f(r)dr} = \frac{1}{c-1} $$.
Therefore, 
\begin{eqnarray*}
  E\left[\frac{h(R_i)f(R_i)}{cg(R_i)-f(R-i)}\right] &=& \int\frac{h(r)f(r)}{cg(r)-f(r)}\frac{cg(r)-f(r)}{c-1} dr \\
  &=& \frac{1}{c-1}\int h(r)f(r)dr \\
  &=& \frac{1}{c-1}E\left[ h(X)\right].
\end{eqnarray*}
Therefore, $E(\hat{E}_2) = E(h(X))$.  So both $\hat{E}_1$ and $\hat{E}_1$ are unbiased estimators of $E(h(X))$.  Also, since the $X_i$'s and $R_i$'s were generated independently, then functions of these random variables will also be indepdendent.  Since $\hat{E}_1$ only involves the $X_i$ and $\hat{E}_2$ only involves the $R_i$, we have that $\hat{E}_1$ and $\hat{E}_2$ are independent. 

\subsection*{(b)}
\begin{eqnarray*}
  var\left[\hat{E}_3(b)\right] &=& var\left[ b\hat{E}_1 + (1-b)\hat{E}_2\right] \\
    &=& b^2 var(\hat{E}_1) + (1-b)^2 var(\hat{E}_2)
\end{eqnarray*}
To find the optimal value of $b$, we differentiate:
\begin{eqnarray*}
  \frac{d}{db}var\left[\hat{E}_3(b)\right] &=& 2b var(\hat{E}_1) - 2(1-b) var(\hat{E}_2)
\end{eqnarray*}
Setting the derivative to zero gives us $$\hat{b}=\frac{var(\hat{E}_2)}{var(\hat{E}_1) + var(\hat{E}_2)}$$
Now, 
\begin{eqnarray*}
  var(\hat{E}_1) &=& var\left(\frac{1}{T}\sum_{i=1}^T h(X_i)\right) \\
  &=& \frac{1}{T^2} \sum_{i=1}^T var(h(X_i)) \\
  &=& \frac{1}{T} var(h(X)).
\end{eqnarray*}

\begin{eqnarray*}
  var(\hat{E}_2) &=& var\left( \frac{1}{n-T} \sum_{i=1}^{n-T} \frac{h(R_i)(c-1)f(R_i)}{cg(R_i)-f(R_i)} \right) \\
  &=& \frac{1}{(n-T)^2}\sum_{i=1}^{n-T}(c-1)^2 var\left(\frac{h(R_i)f(R_i)}{cg(R_i)-f(R_i)}\right) \\
  &=& \frac{1}{(n-T)} \left\{ E\left[ \left( \frac{h(R)f(R)}{cg(R)-f(R)} \right)^2 \right] - \left( E\left[ \frac{h(R)f(R)}{cg(R)-f(R)} \right] \right)^2 \right\}
\end{eqnarray*}
And, we can evaluate:
\begin{eqnarray*}
  E\left[ \left( \frac{h(R)f(R)}{cg(R)-f(R)} \right)^2 \right] &=& \int \frac{h(r)^2f(r)^2}{(cg(r)-f(r))^2}\left( \frac{cg(r)-f(r)}{\sqrt{f(r)}(c-1)} \right)^2 dr \\
  &=& \frac{1}{(c-1)^2}\int h(r)^2 f(r) dr \\
  &=& E(h(X)^2).
\end{eqnarray*}
So, we have that,
\begin{eqnarray*}
  var(\hat{E}_2) &=& \frac{1}{n-T}(E(h(X)^2)-[E(h(X))]^2) \\
  &=& \frac{1}{n-T} var(h(X)).
\end{eqnarray*}
Finally, we can conclude that,
\begin{eqnarray*}
  \hat{b} &=& \frac{var(\hat(E)_2)}{var(\hat(E)_1)+var(\hat(E)_2)} \\
  &=& \frac{var(h(X))/(n-T)}{var(h(X))/T + var(h(X))/(n-T)} \\
  &=& \frac{T}{n}.
\end{eqnarray*}

\end{document}
